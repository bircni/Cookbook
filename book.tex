\documentclass{scrreport}

\usepackage{xcolor}
\usepackage[pdfborder={0 0 0}]{hyperref}

\begin{document}
\pagenumbering{gobble}
\renewcommand{\contentsname}{Inhaltsverzeichnis}
\setcounter{tocdepth}{1}
\tableofcontents
\addtocontents{toc}{~\hfill\textbf{Seite}\par}
\pagebreak
\pagenumbering{arabic}

\chapter{Laugen Brezeln}

\subsection*{Zutaten}

\begin{itemize}
	\item 500g Weizenmehl 550
	\item 225g kaltes Wasser
	\item 40g Butter
	\item 5g Frischehefe
	\item 10g Salz
	\item Hagelsalz zum Bestreuen
	\item 4\%ige Natronlauge
\end{itemize}

\subsection*{Zubereitung}

\begin{enumerate}
	\item Alle Zutaten für ca. 10 Minuten zu einem elastischen Teig verkneten.
	\item 2 Studen bei Raumtemperatur ruhen lassen.
	\item 10 Teiglinge à 80g abwiegen, länglich wirken und dann ohne Mehl auf der Arbeitsfläche zu langen
	      Würsten ausrollen, zu Brezeln einschlagen und ohne Abdeckung für 1 Stunde reifen lassen.
	\item Ofen mit einer metallenen Auflaufform auf 250°C vorheizen.
	\item Eine 4\%ige Natronlauge herstellen und die Brezeln für einige Sekunden darin baden, anschließend
	      zurück aufs Backblech legen. \\\textbf{ACHTUNG:} Natronlauge ist ätzend!
	\item Brezeln einschneiden, mit Salz bestreuen und im vorgeheizten Ofen für 18 Minuten bei 230°C mit etwas
	      Dampf backen.
\end{enumerate}

Source: \href{https://josemola.de/rezepte/brezel}{\textit{\textbf{Jo Semola}}}

\chapter{Frühstücksecken}

\subsection*{Zutaten}

\textbf{Brühstück:}
\begin{itemize}
	\item 100g Wasser 100°C
	\item 30g Altbrot
\end{itemize}
\textbf{Hauptteig:}
\begin{itemize}
	\item alles vom Brühstück
	\item 630g Weizenmehl 550
	\item 350g Wasser
	\item 7g Hefe
	\item 12g Salz
	\item 12g Olivenöl
\end{itemize}

\subsection*{Zubereitung}

\begin{enumerate}
	\item Altbrot mit kochendem Wasser übergießen und abkühlen lassen.
	\item Alle Hauptteig-Zutaten mit Brühstück 6-8 Minuten kneten und abgedeckt 2 Stunden ruhen lassen.
	      Im Abstand von 1 Stunde 2 Mal dehnen und falten.
	\item ca. 2 Stunden bei Raumtemperatur reifen lassen.
	      Währenddessen 2 Mal dehnen und falten.
	\item Für 8-14 Stunden in den Kühlschrank stellen.
	\item Den Backofen auf 250°C vorheizen.
	\item Den Teig schonend auf die bemehlte Arbeitsfläche stürzen, vorsichtig zu einem Rechteck schieben
	      und Dreiecke abstechen. Abgedeckt ruhen lassen bis der Ofen heiß ist.
	\item Teiglinge mit Wasser besprühen oder bestreichen und in den Ofen schieben.
	\item Ordentlich dampfen und die Temperatur direkt auf 230°C reduzieren. Brötchen 20-22 Minuten backen.
\end{enumerate}

Source: \href{https://josemola.de/rezepte/fruehstuecksecken-mit-altbrot-ueber-nacht}{\textit{\textbf{Jo Semola}}}

\end{document}